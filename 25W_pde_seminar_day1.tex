\documentclass[11pt]{article}

\usepackage[utf8]{inputenc}
\usepackage[T1]{fontenc}
\usepackage{amsmath, amssymb, amsthm}
\usepackage{geometry}
\geometry{a4paper, margin=1in}

\theoremstyle{definition}
\newtheorem{definition}{Definition}[section]
\newtheorem{example}{Example}[section]
\theoremstyle{plain}
\newtheorem{proposition}{Proposition}[section]
\newtheorem{remark}{Remark}[section]
\newtheorem{exercise}{Exercise}[section]
\newtheorem{theorem}{Theorem}[section]
\newtheorem{lemma}{Lemma}[section]
\newtheorem{solution}{Solution}[section]

\newcommand{\R}{\mathbb{R}}
\newcommand{\Q}{\mathbb{Q}}
\newcommand{\N}{\mathbb{N}}
\newcommand{\B}{\mathcal{B}}
\newcommand{\Lscr}{\mathcal{L}}
\newcommand{\norm}[1]{\left\| #1 \right\|}
\newcommand{\loc}{{\text{loc}}}
\newcommand{\Wkp}{W^{k,p}}
\newcommand{\Wkploc}{W^{k,p}_{\text{loc}}}
\newcommand{\Wkz}{W^{k,p}_0}
\newcommand{\dx}{dx}
\newcommand{\Hk}{H^{k}}
\newcommand{\Hz}{H^{1}_{0}}
\newcommand{\Hm}{H^{-1}}

\title{PDE Seminar: Day 1 \\ Sobolev Spaces and Distributions}
\author{Yeryang Kang}
\date{January 30, 2026}

\begin{document}

\maketitle


\section{$\sigma$-Algebras and Measurable Sets}

A $\sigma$-algebra provides the formal framework for defining sets that are "measurable" in the sense of measure theory.

\begin{definition}[$\sigma$-algebra]
Let $X$ be a set. A $\sigma$-algebra on $X$ is a non-empty collection $\Sigma$ of subsets of $X$ satisfying:
\begin{enumerate}
    \item $\emptyset \in \Sigma$ and $X \in \Sigma$.
    \item \textbf{Closed under complement:} If $E \in \Sigma$, then $E^c = X \setminus E \in \Sigma$.
    \item \textbf{Closed under countable unions:} If $\{E_k\}_{k=1}^\infty \subseteq \Sigma$, then $\bigcup_{k=1}^\infty E_k \in \Sigma$.
\end{enumerate}
By De Morgan's laws, these conditions imply $\Sigma$ is also closed under countable intersections.
\end{definition}

\section{Set Functions and Measures}

\begin{definition}[Set Function]
A set function is a function whose domain is a family of subsets (usually a $\sigma$-algebra) of a given set $X$, mapping to the extended real line $[-\infty, \infty]$.
\end{definition}

\begin{definition}[Measure]
Let $X$ be a set and $\Sigma$ be a $\sigma$-algebra over $X$. A set function $\mu: \Sigma \to [0, \infty]$ is called a \textbf{positive measure} if:
\begin{enumerate}
    \item $\mu(\emptyset) = 0$.
    \item \textbf{Countable Additivity:} For any sequence $\{E_k\}_{k=1}^\infty$ of pairwise disjoint sets in $\Sigma$:
    \[ 
        \mu\!\left( \bigcup_{k=1}^\infty E_k \right) = \sum_{k=1}^\infty \mu(E_k).
    \]
\end{enumerate}
The triple $(X, \Sigma, \mu)$ is called a \textbf{measure space}.
\end{definition}

\section{Lebesgue Measure on $\R^n$}

The Lebesgue measure extends the notions of length, area, and volume to Euclidean spaces $\R^n$.

\begin{definition}[Lebesgue Outer Measure]
For any subset $E \subset \R^n$, the Lebesgue outer measure $\lambda^*(E)$ is defined as:
\[ 
    \lambda^*(E) := \inf \left\{ \sum_{k=1}^\infty \operatorname{vol}(C_k) \right\},
\]
where the infimum is taken over all sequences of open rectangular cuboids $\{C_k\}$ such that $E \subset \bigcup_{k=1}^\infty C_k$. 
\begin{itemize}
    \item In 1D, $\operatorname{vol}(C_k)$ is the length $l(I) = b-a$.
    \item In $n$-dimensions, for $C = I_1 \times \dots \times I_n$, $\operatorname{vol}(C) = \prod_{i=1}^n l(I_i)$.
\end{itemize}
\end{definition}

\begin{proposition}[Carath\'eodory Criterion]
A set $E \subset \R^n$ is \textbf{Lebesgue measurable} if for every test set $A \subset \R^n$:
\[
    \lambda^*(A) = \lambda^*(A \cap E) + \lambda^*(A \cap E^c).
\]
The collection $\Lscr$ of all such measurable sets forms a $\sigma$-algebra. Restricting $\lambda^*$ to $\Lscr$ gives the Lebesgue measure $m$, i.e., $m(E) = \lambda^*(E)$ for $E \in \Lscr$. The triple $(\R^n, \Lscr, m)$ is the Lebesgue measure space.
\end{proposition}

\begin{exercise}[Properties of Lebesgue Measure]
For $n \geq 1$:
\begin{enumerate}
    \item Verify that $\Lscr$ satisfies the axioms of a $\sigma$-algebra.
    \item Show that the restriction $\lambda^*|_{\Lscr}$ satisfies countable additivity, hence is a measure.
\end{enumerate}
\end{exercise}

\begin{solution}
\ \\
\begin{enumerate}
    \item We only show closure under countable unions. Let $\{E_k\} \subset \Lscr$ be pairwise disjoint and set $E = \bigcup_k E_k$.
    For any $A$, iterating the Carath\'eodory condition gives
    \[
        \lambda^*(A) = \sum_{k=1}^N \lambda^*(A \cap E_k) + \lambda^*\!\left(A \cap \bigcap_{k=1}^N E_k^c\right).
    \]
    Letting $N \to \infty$ yields
    \[
        \lambda^*(A) = \lambda^*(A \cap E) + \lambda^*(A \cap E^c),
    \]
    so $E \in \Lscr$.

    \item For countable additivity, let $\{E_k\} \subset \Lscr$ be disjoint and $E = \bigcup_k E_k$.
    Apply Carath\'eodory with $A = E$:
    \[
        \lambda^*(E) = \sum_{k=1}^N \lambda^*(E_k) + \lambda^*\!\left(E \cap \bigcap_{k=1}^N E_k^c\right).
    \]
    The last term decreases to $0$ as $N \to \infty$, hence
    \[
        \lambda^*(E) = \sum_{k=1}^\infty \lambda^*(E_k).
    \]
    Thus $\lambda^*|_{\Lscr}$ is a measure.
\end{enumerate}
\end{solution}

\begin{example}[Dirichlet Function in 1D]
Define $f: \R \to \R$ as the indicator of the rationals:
\[ 
    f(x) = \mathbf{1}_{\Q}(x) = 
    \begin{cases} 
        1 & \text{if } x \in \Q, \\ 
        0 & \text{if } x \in \R \setminus \Q.
    \end{cases}
\]
On $[0,1]$:
\begin{itemize}
    \item $\Q \cap [0,1]$ is countable, so $m(\Q \cap [0,1]) = 0$.
    \item $m([0,1] \setminus \Q) = 1$.
\end{itemize}
The Lebesgue integral is
\[ 
    \int_{[0,1]} f \, dm = 1 \cdot m(\Q) + 0 \cdot m(\R \setminus \Q) = 0.
\]
\end{example}


\section{Lebesgue Spaces $L^p$}

Let $(X, \Sigma, \mu)$ be a measure space and $f$ a measurable function.

\subsection{$L^p$ spaces for $0 < p < \infty$}
Define the $L^p$ (semi-)norm by
\[ 
    \norm{f}_p := \left( \int_X |f|^p \, d\mu \right)^{1/p}.
\]
The space $L^p(X, \mu)$ consists of all measurable $f$ with $\norm{f}_p < \infty$ (identifying functions equal a.e.).

\subsection{$L^\infty$ space}
For a measurable $g: X \to [0, \infty]$, define the set of essential bounds
\[
    S = \{ \alpha \in [0, \infty] \mid \mu(\{x : g(x) > \alpha\}) = 0 \}.
\]
If $S = \emptyset$, set $\norm{g}_\infty = \infty$; otherwise, the \textbf{essential supremum} is
\[
    \norm{g}_\infty := \inf S.
\]
Functions in $L^\infty(X, \mu)$ are essentially bounded.


\section{Key Inequalities and Convexity}


\subsection{Conjugate Exponents}
\begin{definition}[Conjugate Exponents]
Two positive real numbers $p$ and $q$ are called \textbf{conjugate exponents} if
\[
    \frac{1}{p} + \frac{1}{q} = 1.
\]
Common pairs: $p=q=2$ (self‑conjugate, Hilbert space case) and $p=1, q=\infty$ (limiting case).
\end{definition}

\subsection{Hölder's Inequality}
\begin{proposition}[Basic Hölder Inequality]
Let $p, q$ be conjugate exponents with $1 < p < \infty$, and let $f, g$ be non‑negative measurable functions on $(X, \mu)$. Then
\[
    \int_X f g \, d\mu \leq \left( \int_X f^p \, d\mu \right)^{1/p} \left( \int_X g^q \, d\mu \right)^{1/q}.
\]
\end{proposition}

\begin{proof}
Set $A = \|f\|_p$, $B = \|g\|_q$.
\begin{enumerate}
    \item If $A = 0$, then $f = 0$ a.e., so $fg = 0$ a.e., and the inequality holds.
    \item If $A > 0$ and $B = \infty$, the inequality is trivial.
    \item Assume $0 < A, B < \infty$. Define normalized functions $F = f/A$, $G = g/B$, so that $\|F\|_p = \|G\|_q = 1$.
    
    For any $x$ with $0 < F(x), G(x) < \infty$, write $F(x) = e^{s/p}$, $G(x) = e^{t/q}$. By convexity of the exponential,
    \[
        e^{\frac{s}{p} + \frac{t}{q}} \leq \frac{1}{p} e^s + \frac{1}{q} e^t,
    \]
    which gives the pointwise Young inequality
    \[
        F(x) G(x) \leq \frac{F(x)^p}{p} + \frac{G(x)^q}{q} \qquad \text{for a.e. } x. \tag{*}
    \]
    Integrating (*) over $X$ yields
    \[
        \int_X F G \, d\mu \leq \frac{1}{p} \int_X F^p \, d\mu + \frac{1}{q} \int_X G^q \, d\mu = \frac{1}{p} + \frac{1}{q} = 1.
    \]
    Substituting back $F = f/A$, $G = g/B$ gives $\frac{1}{AB} \int_X fg \, d\mu \leq 1$, i.e., $\int_X fg \, d\mu \leq A B$.
\end{enumerate}
\end{proof}

\subsection{Convex Functions}
\begin{definition}[Convex Function]
A function $\phi: (a, b) \to \R$ (with $-\infty \leq a < b \leq \infty$) is \textbf{convex} if
\[
    \phi((1-\lambda)x + \lambda y) \leq (1-\lambda)\phi(x) + \lambda\phi(y)
\]
for all $x, y \in (a, b)$ and all $\lambda \in [0,1]$.
\end{definition}

\subsection{General Hölder Inequality}
\begin{proposition}[Hölder in $L^p$]
Let $1 \leq p \leq \infty$ and $q$ be its conjugate exponent. If $f \in L^p(\mu)$ and $g \in L^q(\mu)$, then $fg \in L^1(\mu)$ and
\[
    \| fg \|_1 \leq \| f \|_p \, \| g \|_q.
\]
\end{proposition}

\begin{proof}
The case $1 < p < \infty$ follows from the basic Hölder inequality applied to $|f|$ and $|g|$.

For $p = \infty$ (so $q = 1$), the definition of $\| \cdot \|_\infty$ gives $|f(x)g(x)| \leq \|f\|_\infty |g(x)|$ a.e. Integrating yields the result. The case $p=1, q=\infty$ is symmetric.
\end{proof}


\section{Weak Derivatives}


\subsection{Motivation and Definition}
We begin by substantially weakening the notion of partial derivatives.

\noindent \textbf{Notation.} Let $C_c^\infty(U)$ denote the space of infinitely differentiable functions $\phi: U \to \mathbb{R}$ with compact support in $U$. Functions $\phi \in C_c^\infty(U)$ are called \emph{test functions}.

\noindent \textbf{Motivation.} Assume $u \in C^1(U)$. For any $\phi \in C_c^\infty(U)$, integration by parts gives
\begin{equation}
    \int_U u \phi_{x_i} \, dx = -\int_U u_{x_i} \phi \, dx \qquad (i=1,\dots,n). \tag{1}
\end{equation}
There are no boundary terms because $\phi$ has compact support. More generally, if $u \in C^k(U)$ and $\alpha = (\alpha_1,\dots,\alpha_n)$ is a multi‑index of order $|\alpha| = k$, then
\begin{equation}
    \int_U u \, D^\alpha \phi \, dx = (-1)^{|\alpha|} \int_U D^\alpha u \, \phi \, dx. \tag{2}
\end{equation}
This follows by applying (1) repeatedly.

Observing that the left‑hand side of (2) makes sense for any locally integrable $u$, we are led to the following definition.

\begin{definition}[Weak Derivative]
Let $u, v \in L^1_{\loc}(U)$ and let $\alpha$ be a multi‑index. We say that $v$ is the $\alpha^{\text{th}}$ \textbf{weak partial derivative} of $u$, written
\[
    D^\alpha u = v,
\]
provided
\begin{equation}
    \int_U u \, D^\alpha \phi \, dx = (-1)^{|\alpha|} \int_U v \phi \, dx \qquad \text{for all } \phi \in C_c^\infty(U). \tag{3}
\end{equation}
\end{definition}

\begin{lemma}[Uniqueness]
A weak $\alpha^{\text{th}}$-partial derivative of $u$, if it exists, is unique up to sets of measure zero.
\end{lemma}

\begin{proof}
Assume $v, \tilde{v} \in L^1_{\loc}(U)$ both satisfy (3) for all test functions $\phi$. Then
\[
    \int_U (v - \tilde{v}) \phi \, dx = 0 \quad \forall \phi \in C_c^\infty(U),
\]
which implies $v = \tilde{v}$ almost everywhere.
\end{proof}

\subsection{Example}
\begin{example}
Let $n=1$, $U = (0,2)$, and
\[
    u(x) = 
    \begin{cases}
        x, & 0 < x \leq 1, \\
        1, & 1 \leq x < 2.
    \end{cases}
\]
Define
\[
    v(x) = 
    \begin{cases}
        1, & 0 < x \leq 1, \\
        0, & 1 < x < 2.
    \end{cases}
\]
We claim $u' = v$ in the weak sense. Indeed, for any $\phi \in C_c^\infty(U)$,
\begin{align*}
    \int_0^2 u \phi' \, dx 
    &= \int_0^1 x \phi' \, dx + \int_1^2 \phi' \, dx \\
    &= \big[ x\phi(x) \big]_0^1 - \int_0^1 \phi \, dx + \phi(2) - \phi(1) \\
    &= -\int_0^1 \phi \, dx = -\int_0^2 v \phi \, dx,
\end{align*}
which is exactly condition (3) with $\alpha = 1$.
\end{example}


\section{Compact Sets in $\R^n$: A Brief Review}

\begin{itemize}
\item Typical examples of compact sets in $\mathbb{R}^n$ (by Heine--Borel theorem):
\begin{itemize}
\item Finite sets.
\item Closed hypercubes: $[a_1,b_1] \times \cdots \times [a_n,b_n]$.
\item Closed balls: $\overline{B}(x_0,r)$.
\item $n$-spheres: $\{ x \in \mathbb{R}^{n+1} \mid |x| = 1 \}$.
\end{itemize}
\end{itemize}


\section{Sobolev Spaces}

Fix $1 \leq p \leq \infty$ and let $k$ be a nonnegative integer. Sobolev spaces are function spaces whose members have weak derivatives up to order $k$ lying in $L^p$.

\subsection{Definition and Basic Properties}
\begin{definition}[Sobolev Space $\Wkp(U)$]
The Sobolev space $\Wkp(U)$ consists of all locally summable functions $u : U \rightarrow \mathbb{R}$ such that for each multi-index $\alpha$ with $|\alpha| \leq k$, the weak derivative $D^{\alpha}u$ exists and belongs to $L^p(U)$.
\end{definition}

\begin{remark} 
\mbox{}\\

(i) When $p = 2$, we denote $\Hk(U) = \Wkp(U)$ for $k = 0, 1, \ldots$. The letter $H$ is used because $\Hk(U)$ is a Hilbert space. Note that $H^{0}(U) = L^2(U)$.
\bigskip

(ii) We identify functions in $\Wkp(U)$ that agree almost everywhere.
\end{remark}

\begin{definition}[Sobolev Norm]
For $u \in \Wkp(U)$, define its Sobolev norm by
\[
\norm{u}_{\Wkp(U)} := 
\begin{cases}
\left(\displaystyle\sum_{|\alpha| \leq k} \int_U |D^{\alpha}u|^p \dx\right)^{1/p} & (1 \leq p < \infty), \\[12pt]
\displaystyle\sum_{|\alpha| \leq k} \operatorname*{ess\,sup}_U |D^{\alpha}u| & (p = \infty).
\end{cases}
\]
\end{definition}

\subsection{Convergence and Local Spaces}
\begin{definition}[Convergence in Sobolev Spaces]
\mbox{}\\
\begin{enumerate}
\item Let $\{u_m\}_{m=1}^{\infty}, u \in \Wkp(U)$. We say $u_m$ \textbf{converges to $u$ in $\Wkp(U)$}, written
\[
u_m \to u \quad \text{in } \Wkp(U),
\]
provided $\displaystyle\lim_{m\to\infty} \norm{u_m - u}_{\Wkp(U)} = 0$.

\item We say $u_m \to u$ \textbf{in $\Wkploc(U)$} if $u_m \to u$ in $\Wkp(V)$ for every $V \subset\subset U$ (i.e., $V$ is compactly contained in $U$).
\end{enumerate}
\end{definition}

\subsection{Space $\Wkz(U)$: Functions with Zero Boundary Values}
\begin{definition}[$\Wkz(U)$]
We denote by $\Wkz(U)$ the closure of $C^{\infty}_{c}(U)$ in $\Wkp(U)$. 
\end{definition}

Thus $u \in \Wkz(U)$ if and only if there exist functions $u_m \in C^{\infty}_{c}(U)$ such that $u_m \to u$ in $\Wkp(U)$. Intuitively, $\Wkz(U)$ consists of functions $u \in \Wkp(U)$ that satisfy
\[
``D^{\alpha}u = 0 \text{ on } \partial U" \quad \text{for all } |\alpha| \leq k-1,
\]
in a weak sense. This interpretation will be made precise by the theory of traces.


\section{Poincaré Inequality}

The Poincaré inequality shows that for functions vanishing on the boundary, the $L^p$ norm can be controlled by the norm of the gradient.

\begin{theorem}[Poincaré Inequality for $W^{1,p}_0(\Omega)$]
Let $\Omega$ be a bounded domain. If $u \in W^{1,p}_0(\Omega)$, there exists a constant $C = C(\Omega, p)$ such that
\[
\norm{u}_{L^p(\Omega)} \leq C \norm{\nabla u}_{L^p(\Omega)}.
\]
\end{theorem}

\begin{proof}
We present the proof for $n=1$; the generalization to $n \geq 2$ is straightforward.

First, prove the inequality for $u \in C_0^\infty(\Omega)$. Since $\Omega$ is bounded, enclose it in a cube $Q = [0, a]^n$. Extend $u$ by zero to $Q$, so $u \in C_0^\infty(Q)$.

For $x = (x_1, \dots, x_n) \in Q$, we have
\[
u(x) = \int_0^{x_1} \partial_1 u(t, x_2, \dots, x_n) \, dt,
\]
because $u(0, x_2, \dots, x_n) = 0$. By Hölder's inequality,
\[
|u(x)|^p \leq \left( \int_0^{x_1} |\partial_1 u(t, x_2, \dots, x_n)| \, dt \right)^p 
\leq x_1^{p-1} \int_0^{x_1} |\partial_1 u(t, x_2, \dots, x_n)|^p \, dt.
\]

Integrate over $Q$:
\begin{align*}
\int_Q |u(x)|^p \dx 
&\leq \int_Q x_1^{p-1} \left( \int_0^{x_1} |\partial_1 u(t, x')|^p \, dt \right) \dx \\
&= \int_{[0,a]^{n-1}} \int_0^a x_1^{p-1} \left( \int_0^{x_1} |\partial_1 u(t, x')|^p \, dt \right) dx_1 \, dx',
\end{align*}

where $x' = (x_2, \dots, x_n)$. Changing the order of integration,
\begin{align*}
\int_0^a x_1^{p-1} & \left( \int_0^{x_1} |\partial_1 u(t, x')|^p \, dt \right) dx_1 \\
&= \int_0^a |\partial_1 u(t, x')|^p \left( \int_t^a x_1^{p-1} \, dx_1 \right) dt \\
&= \int_0^a |\partial_1 u(t, x')|^p \cdot \frac{a^p - t^p}{p} \, dt \\
&\leq \frac{a^p}{p} \int_0^a |\partial_1 u(t, x')|^p \, dt.
\end{align*}

Therefore,
\[
\int_Q |u|^p \dx \leq \frac{a^p}{p} \int_Q |\partial_1 u|^p \dx 
\leq \frac{a^p}{p} \int_Q |\nabla u|^p \dx.
\]
Hence,
\[
\norm{u}_{L^p(\Omega)} = \norm{u}_{L^p(Q)} \leq \frac{a}{p^{1/p}} \norm{\nabla u}_{L^p(\Omega)}.
\]

Now for general $u \in W^{1,p}_0$, take a sequence $\{u_k\} \subset C_0^\infty(\Omega)$ with $u_k \to u$ in $W^{1,p}(\Omega)$. Applying the inequality to each $u_k$ and passing to the limit yields the result.
\end{proof}

\begin{remark}[General Poincaré Inequality]
For $u \in W_0^{n,p}(\Omega)$ with $\Omega$ bounded, there exists $C = C(\Omega, n, p)$ such that
\[
\norm{u}_{L^p(\Omega)} \leq C \sum_{|\alpha| = n} \norm{D^\alpha u}_{L^p(\Omega)}.
\]
\end{remark}


\section{Boundary Values and Trace Operator}

\begin{remark}
Boundary values of Sobolev functions are defined via a \textbf{trace operator}. For $u \in W_0^{1,p}(\Omega)$, the trace $T u$ is a well-defined element of $L^p(\partial\Omega)$ (for sufficiently smooth $\partial\Omega$). This is not a pointwise restriction but a continuous linear operator
\[
T : W^{1,p}(\Omega) \to L^p(\partial\Omega).
\]
The space $W^{1,p}(\Omega)$ precisely consists of functions whose trace is zero.
\end{remark}


\section{Distributions and Weak Solutions}


\subsection{Distributions: Basic Idea}
A \textbf{distribution} generalizes the concept of a function. Formally, a distribution $T$ is a continuous linear functional
\[
T : C_0^\infty(\Omega) \to \mathbb{R}.
\]
\begin{example}[Regular Distributions]
If $g \in L^1_{\loc}(\Omega)$, then
\[
T_g(\varphi) := \int_\Omega g(x)\varphi(x)\dx
\]
defines a distribution.
\end{example}
\begin{example}[Dirac Delta]
The Dirac delta $\delta_{x_0}$ is defined by $\delta_{x_0}(\varphi)=\varphi(x_0)$.
\end{example}

Test functions $\varphi \in C_0^\infty(\Omega)$ are ideal because they are smooth and vanish near the boundary, so integration by parts produces no boundary terms.

\subsection{Weak Formulation of Poisson's Equation}
Consider the Dirichlet problem
\[
-\Delta u = f \quad \text{in } \Omega, \qquad u=0 \text{ on } \partial\Omega.
\]
If $u$ were smooth, multiplying by a test function $\varphi \in C_0^\infty(\Omega)$ and integrating by parts gives
\[
\int_\Omega \nabla u \cdot \nabla \varphi \dx = \int_\Omega f\varphi \dx.
\]
This identity makes sense even when $u$ is not twice differentiable.

\begin{definition}[Weak Solution]
A function $u \in \Hz(\Omega)$ is a \textbf{weak solution} of $-\Delta u = f$ with $u=0$ on $\partial\Omega$ if
\[
\int_\Omega \nabla u \cdot \nabla \varphi \dx = \int_\Omega f\varphi \dx
\quad \forall \varphi \in C_0^\infty(\Omega).
\]
\end{definition}

\textbf{Key idea:} Derivatives are transferred from $u$ to the test function $\varphi$. This allows solutions when $u$ lacks classical second derivatives and when $f$ is only in $L^2(\Omega)$ or even $\Hm(\Omega)$.

\subsection{The Dual Space $H^{-1}(\Omega)$}
\begin{definition}
The space $\Hm(\Omega)$ is the dual space of $\Hz(\Omega)$:
\[
\Hm(\Omega) = \big( \Hz(\Omega) \big)^*.
\]
Its elements are continuous linear functionals on $\Hz(\Omega)$.
\end{definition}

\begin{remark}[Interpretation]
\begin{align*}
\bullet\;& \text{Elements of } \Hm(\Omega) \text{ are distributions that act continuously on } \Hz(\Omega).\\
\bullet\;& \text{The negative index indicates that } H^{-1} \text{ functions are ``one derivative less regular'' than } L^2.\\
\bullet\;& -\Delta : \Hz(\Omega) \to \Hm(\Omega) \text{ is the natural mapping of the Laplacian}.\\
\bullet\;& \text{In the weak formulation, it is natural to allow } f \in \Hm(\Omega).
\end{align*}
\end{remark}


\textbf{Key takeaway:} $\Hm(\Omega)$ is the natural space for the right-hand side $f$ in the weak formulation of second-order elliptic equations, ensuring the formulation remains well-defined even for irregular data.

\end{document}