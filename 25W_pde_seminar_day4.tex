\documentclass[a4paper,11pt]{article}

\usepackage{amsmath, amssymb, amsthm}
\usepackage{geometry}
\usepackage{hyperref}
\usepackage{xcolor}

\geometry{left=25mm, right=25mm, top=30mm, bottom=30mm}

\newtheorem{theorem}{Theorem}[section]
\newtheorem{lemma}[theorem]{Lemma}
\newtheorem{definition}[theorem]{Definition}
\newtheorem{remark}[theorem]{Remark}

\title{PDE Seminar: Day 4 \\ Schauder Estimates}
\author{Yeryang Kang}
\date{February 9, 2026}

\begin{document}

\maketitle

\section{Overview: Classical Solvability}
In the previous section, we established the uniqueness of solutions using Maximum Principles. Now, we address the question of "existence" and "regularity" for the Dirichlet problem:
\begin{equation} \label{eq:dirichlet}
    \begin{cases}
    Lu = f & \text{in } \Omega \\
    u = g & \text{on } \partial\Omega
    \end{cases}
\end{equation}

The space $C^k$ is not complete, which implies that the limit of a sequence of $C^k$ functions may not belong to the same class. To resolve this issue, we introduce H\"older spaces. Specifically, since $C^{k,\alpha}$ lies between $C^k$ and $C^{k+1}$, the additional regularity requirement is strictly less than one full derivative; nevertheless, the functional analytic properties of the space are significantly superior to those of $C^k$. This is precisely the reason why we introduce H\"older spaces. In particular, $C^{k,\alpha}$ is a Banach space (a complete normed space).


\section{Theorem 3.4: Schauder Estimates}

The Schauder estimates are \textit{a priori} estimates that bound the $C^{2,\alpha}$ norm of the solution by the $C^{0,\alpha}$ norm of the source term $f$.

\begin{theorem}[6.2. Schauder Estimates]
Let $u \in C^{2,\alpha}(\Omega)$ satisfy $Lu = f$ with $f \in C^\alpha$. Under uniform ellipticity and $C^\alpha$ bounds on coefficients:
\begin{equation}
    |u|^*_{2,\alpha;\Omega} \leq C(|u|_{0;\Omega} + |f|^{(2)}_{0,\alpha;\Omega})
\end{equation}
where $C = C(n, \alpha, \lambda, \Lambda)$.
\end{theorem}

\begin{proof}[Sketch of Proof]
The proof generalizes the regularity theory of the Laplacian to variable coefficient operators via the \textbf{"Freezing of Coefficients"} technique.
\begin{enumerate}
    \item \textbf{Base Case (Constant Coefficients):} We assume the estimate holds for operators with constant coefficients (e.g., $L_0 = \sum a_{ij}(x_0) D_{ij}$). This is derived from the Newtonian potential theory.
    \item \textbf{Localization:} We focus on a small ball $B_r(x_0)$ inside the domain using a smooth cutoff function $\eta$.
    \item \textbf{Perturbation:} Inside this small ball, the variable coefficients $a_{ij}(x)$ deviate very little from the constant value $a_{ij}(x_0)$. We treat the difference $(L - L_0)u$ as a small perturbation error.
    \item \textbf{Absorption:} We show that the "error" terms can be absorbed into the left-hand side of the inequality because their norm is proportional to the small radius $r^\alpha$.
    \item \textbf{Interpolation:} Finally, we use interpolation inequalities to control intermediate derivatives (like $\nabla u$) using only the highest derivatives ($D^2 u$) and the function value ($u$).
\end{enumerate}
\end{proof}

\begin{proof}[Detailed Proof (Simplified)]
We proceed in four clear steps: Localization, Freezing, Estimation, and Absorption.

\textbf{Step 1: Localization.}
Fix a point $x_0 \in \Omega'$ and choose a radius $r$ small enough so that the ball $B_r(x_0) \subset \Omega$.
Let $\eta \in C_c^\infty(B_r)$ be a cutoff function such that $\eta \equiv 1$ in $B_{r/2}$ and $0 \le \eta \le 1$.
Define $v = \eta u$. Note that $v = u$ in the smaller ball $B_{r/2}$.

\textbf{Step 2: Freezing the Coefficients.}
We compare the operator $L$ (with variable coefficients $a_{ij}(x)$) to the "frozen" operator $L_0$ (with constant coefficients $a_{ij}(x_0)$):
\begin{equation*}
    L_0 = \sum_{i,j} a_{ij}(x_0) D_{ij}.
\end{equation*}
We want to apply the known Schauder estimate for $L_0$ to $v$. Let's compute $L_0 v$:
\begin{align*}
    L_0 v &= L_0(\eta u) \\
          &= \eta L_0 u + \text{Commutators involving } \nabla\eta, \nabla u \dots \\
          &= \eta (L_0 - L)u + \eta Lu + \text{Lower Order Terms}.
\end{align*}
Since $Lu = f$, we can rearrange this as:
\begin{equation} \label{eq:frozen}
    L_0 v = \eta f + \eta \sum_{i,j} (a_{ij}(x_0) - a_{ij}(x)) D_{ij} u + \text{LOT},
\end{equation}
where $\text{LOT}$ (Lower Order Terms) contains terms with $D u$ and $u$, which are "easier" to bound.

\textbf{Step 3: Applying the Constant Coefficient Estimate.}
We apply the standard Schauder estimate for constant coefficients to equation (\ref{eq:frozen}):
\begin{equation*}
    \|v\|_{C^{2,\alpha}} \le C \left( \| \text{RHS of } (\ref{eq:frozen}) \|_{C^{0,\alpha}} \right).
\end{equation*}
Let's analyze the critical term on the RHS: $E = \eta \sum (a_{ij}(x_0) - a_{ij}(x)) D_{ij} u$.
The $C^{0,\alpha}$ norm of a product roughly satisfies $\|gh\|_\alpha \le \|g\|_\alpha \|h\|_\infty + \|g\|_\infty \|h\|_\alpha$.
The key insight is that on the support of $\eta$ (which is $B_r$), the difference in coefficients is small:
\begin{equation*}
    |a_{ij}(x_0) - a_{ij}(x)| \le [a_{ij}]_\alpha |x_0 - x|^\alpha \le [a_{ij}]_\alpha r^\alpha.
\end{equation*}
Thus, the norm of the error term is bounded by:
\begin{equation*}
    \|E\|_{C^{0,\alpha}} \le C r^\alpha \|u\|_{C^{2,\alpha}(B_r)} + C_r \|u\|_{C^2(B_r)}.
\end{equation*}

\textbf{Step 4: Absorption and Conclusion.}
Substituting the bound for $E$ back into the main estimate:
\begin{equation*}
    \|u\|_{C^{2,\alpha}(B_{r/2})} \le \|v\|_{C^{2,\alpha}} \le C \left( \|f\|_{C^{0,\alpha}} + r^\alpha \|u\|_{C^{2,\alpha}(B_r)} + C_r \|u\|_{C^1} \right).
\end{equation*}
Notice that $\|u\|_{C^{2,\alpha}}$ appears on the Right-Hand Side. However, it is multiplied by $r^\alpha$. By choosing the radius $r$ \textbf{sufficiently small} (depending on the ellipticity and coefficients), we can ensure $C r^\alpha \le 1/2$.
We can then subtract this term to the Left-Hand Side ("absorbing" the highest order term):
\begin{equation*}
    \frac{1}{2} \|u\|_{C^{2,\alpha}(B_{r/2})} \le C \left( \|f\|_{C^{0,\alpha}} + \|u\|_{C^1(B_r)} \right).
\end{equation*}
Finally, using the interpolation inequality $\|u\|_{C^1} \le \epsilon \|u\|_{C^{2,\alpha}} + C_\epsilon \|u\|_{C^0}$ to handle the lower order terms, and covering the compact set $\Omega'$ with finitely many such small balls, we arrive at the desired result:
\begin{equation*}
    \|u\|_{C^{2,\alpha}(\Omega')} \le C \left( \|f\|_{C^{0,\alpha}(\Omega)} + \|u\|_{C^0(\Omega)} \right).
\end{equation*}
\end{proof}



\section{Theorem 3.5: Method of Continuity}

Schauder estimates provide a bound "if" a solution exists. The Method of Continuity uses this bound to prove that a solution "actually" exists.

\begin{theorem}[Method of Continuity] \label{thm:continuity}
Let $\Omega$ be $C^{2,\alpha}$ and $L$ be strictly elliptic with $c \le 0$. If the Dirichlet problem for $\Delta$ is solvable for all $C^\alpha$ data, then $Lu = f, u = \varphi$ is uniquely solvable in $C^{2,\alpha}(\overline{\Omega})$.
\end{theorem}

\begin{proof}[Sketch of Proof]
Define a family of operators $L_t = (1-t)\Delta + tL$. 
\begin{enumerate}
    \item Let $E = \{t \in [0,1] : L_t u = f \text{ is solvable}\}$. 
    \item $0 \in E$ by assumption. $E$ is shown to be open via the Inverse Function Theorem (or Banach Fixed Point Theorem).
    \item $E$ is shown to be closed using the a priori Schauder estimates to pass to the limit. Thus $1 \in E$.
\end{enumerate}
\end{proof}






\begin{proof}[Detailed Proof]
The proof relies on establishing estimates for the Laplacian (Model Case) and extending the result to general operators using the Method of Continuity.

\textbf{Step 1: The Model Case ($\Delta u = f$).}
The solution is represented by the Newtonian potential $u = N * f$. The crucial difficulty is that the kernel for the second derivatives, $D^2 N(z)$, behaves like $|z|^{-n}$, which is not integrable near the origin.
However, we exploit the H\"older continuity of $f$. We can express the second derivative roughly as:
\begin{equation*}
    D^2 u(x) \sim \int_{\mathbb{R}^n} D^2 N(x-y) \left( f(y) - f(x) \right) \, dy.
\end{equation*}
Since $f \in C^{0,\alpha}$, the difference term provides a decay factor: $|f(y) - f(x)| \le C |x-y|^\alpha$.
Multiplying the singularity $|x-y|^{-n}$ by this factor yields $|x-y|^{-n+\alpha}$. Since $-n+\alpha > -n$, this new kernel \textit{is} locally integrable. This "cancellation of singularity" ensures that $D^2 u$ exists and belongs to $C^{0,\alpha}$.

\textbf{Step 2: The Method of Continuity.}
We connect the Laplacian to the general operator $L$ using the family $L_t = (1-t)\Delta + tL$ for $t \in [0,1]$. Let $E$ be the set of $t$ for which $L_t u = f$ is solvable.
\begin{itemize}
    \item \textbf{Non-empty ($0 \in E$):} We know the Laplacian ($t=0$) is solvable from Step 1.
    \item \textbf{Openness:} If $L_{t_0}$ is invertible, standard operator theory implies that small perturbations (nearby $t$) are also invertible. Thus, $E$ is open.
    \item \textbf{Closedness:} This is where the Schauder estimate is vital. If a sequence $t_k \in E$ converges to $t^*$, the corresponding solutions $u_k$ satisfy the uniform bound $\|u_k\|_{C^{2,\alpha}} \le C \|f\|_{C^{0,\alpha}}$. By compactness (Arzel\`a-Ascoli), the sequence $u_k$ converges to a limit solution $u^*$, proving that $t^* \in E$.
\end{itemize}
Since $E$ is non-empty, open, and closed in $[0,1]$, we must have $E=[0,1]$. Therefore, the problem is solvable for $L$ (at $t=1$).
\end{proof}


\subsection{Why Hölder Spaces? (Proof Idea)}
The proof is built upon the Newtonian potential for the Laplacian.
\begin{enumerate}
    \item \textbf{Constant Coefficients (Model Case):} Consider $\Delta u = f$. The solution is given by convolution with the fundamental solution $N(x)$:
    \[ u(x) = \int_{\mathbb{R}^n} N(x-y) f(y) dy \]
    Taking two derivatives yields a singular integral:
    \[ D_{ij}u(x) = \int_{\mathbb{R}^n} D_{ij}N(x-y) f(y) dy \]
    The kernel $D_{ij}N$ is not integrable near the singularity. However, using the Hölder condition $|f(y) - f(x)| \le K|x-y|^\alpha$, we can utilize the "cancellation of singularities" property. This ensures that $D^2 u$ exists and is Hölder continuous.
    
    \item \textbf{Variable Coefficients (Freezing Coefficients):} For the general operator $Lu$, we fix a point $x_0$ and rewrite the equation as:
    \[ L_{x_0} u = f + (L_{x_0} - L)u \]
    where $L_{x_0}$ has constant coefficients frozen at $x_0$. Since the coefficients are continuous, the error term $(L_{x_0} - L)$ is small in a small neighborhood of $x_0$. We can then view this as a perturbation of the constant coefficient case.
\end{enumerate}


\section{Summary of Chapter 6}
\begin{itemize}
    \item Schauder Estimates (Thm 3.4): Provide the necessary \textit{a priori} bounds ($C^{2,\alpha}$ control).
    \item Method of Continuity (Thm 3.5): Converts these bounds into an existence proof by deforming a known operator (Laplacian) into the target operator.
    \item This establishes the "Classical Existence Theory" for elliptic PDEs.
\end{itemize}

\end{document}